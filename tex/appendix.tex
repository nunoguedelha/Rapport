
\appendix

\chapter{Notations en Algèbre spatiale} \label{appx_notations}

\chapter{Quelques démonstrations en algèbre spatiale} \label{appx_dem}

\section{Invariance du vecteur spacial $\widehat{v}$}

\subsection{invariance entre deux bases de \emph{Plücker} ayant la même orientation}

Nous allons démontrer l'invariance du vecteur spatial dans le cas simple illustré ci-dessous:

\dispThreeFig[H]
{4}{Deux bases de \emph{Plücker} de même orientation.}
{5}{Expression de $\textbf{d}_{Py}$ et $\textbf{d}_{Pz}$ en $O$.}
{6}{cas de $\textbf{d}_{Pz}$ (projection suivant $\textbf{d}_{z}$).}
{Expression de $\textbf{d}_{Px}$, $\textbf{d}_{Py}$ et $\textbf{d}_{Pz}$ en $O$.}
{vitessePointCoincidant}

On considère les deux bases de \emph{Plücker}:

\begin{align*}
D_{O} = \lbrace &\textbf{d}_{Ox}, \textbf{d}_{Oy}, \textbf{d}_{Oy}, \textbf{d}_{x}, \textbf{d}_{y}, \textbf{d}_{z} \rbrace \\
D_{P} = \lbrace &\textbf{d}_{Px}, \textbf{d}_{Py}, \textbf{d}_{Py}, \textbf{d}_{x}, \textbf{d}_{y}, \textbf{d}_{z} \rbrace \\
\end{align*}

Dans la section \ref{algSpa_Vitesse}, figure \ref{fig_transPlucker}, nous avons montré la relation entre les vecteurs unitaires de mouvement des deux bases de \emph{Plücker} $D_{O}$ et $D_{P}$:

\begin{align*}
  \textbf{d}_{Px} &= \textbf{d}_{Ox} \\
  \textbf{d}_{Py} &= \textbf{d}_{Oy}+r \textbf{d}_{z} \\
  \textbf{d}_{Pz} &= \textbf{d}_{Oz}-r \textbf{d}_{y}
\end{align*}

On considère à présent un corps $C$ en translation et en rotation autour d'un axe quelconque dans l'espace. Le vecteur spatial $\widehat{v}$ de ce corps en $O$, s'exprime dans la base de \emph{Plücker} $D_{O}$ comme suit:

\begin{equation*}
  \widehat{v} = w_{x}\textbf{d}_{Ox} + w_{y}\textbf{d}_{Oy} + w_{z}\textbf{d}_{Oz} + v_{Ox}\textbf{d}_{x} + v_{Oy}\textbf{d}_{y} + v_{Oz}\textbf{d}_{z}
\end{equation*}

Le vecteur spatial $\widehat{v'}$ de ce corps en $P$, s'exprime dans la base de \emph{Plücker} $D_{P}$ comme suit:

\begin{equation*}
  \widehat{v'} = w_{x}\textbf{d}_{Px} + w_{y}\textbf{d}_{Py} + w_{z}\textbf{d}_{Pz} + v_{Px}\textbf{d}_{x} + v_{Py}\textbf{d}_{y} + v_{Pz}\textbf{d}_{z}
\end{equation*}

Nous voulons l'exprimer dans la base $D_{O}$. $w_{x}$, $w_{y}$ et $w_{z}$ sont connus. Calculons $v_{Px}$, $v_{Py}$ et $v_{Pz}$:

\minipages[2]{.4}{.6}{}
{%
\begin{alignat*}{3}
  & &&\phantom{xxx} v_{P} &&= v_{O} + w \times \overrightarrow{OP} \\
  \vspace{0.5cm}
  &\iff &&\phantom{xxx} \underline{v}_{P}
  &&=
  \underline{v}_{O}+\underline{w}\times
  \begin{bmatrix}
    r\\
    0\\
    0
  \end{bmatrix} \\
  \vspace{0.5cm}
  &\iff
  &&\begin{bmatrix}
    v_{Px} \\
    v_{Py} \\
    v_{Pz}
  \end{bmatrix}
  &&=
  \begin{bmatrix}
    v_{Ox}\\
    v_{Oy}+w_{z}r\\
    v_{Oz}-w_{y}r
  \end{bmatrix}
  \quad \texttt{or} \quad
  \widehat{\underline{v}'}
  =
  \begin{bmatrix}
    \underline{w} \\
    \underline{v}_{P}
  \end{bmatrix}
\end{alignat*}
}
{%
\begin{alignat*}{2}
  &\iff
  \widehat{v'} &&= w_{x}\textbf{d}_{Px} + w_{y}\textbf{d}_{Py} + w_{z}\textbf{d}_{Pz} \\
  &            &&\phantom{{}={}} + v_{Px}\textbf{d}_{x} + v_{Py}\textbf{d}_{y} + v_{Pz}\textbf{d}_{z} \\
  &            &&= w_{x}\textbf{d}_{Ox} + w_{y}(\textbf{d}_{Oy}+r\textbf{d}_{z}) + w_{z}(\textbf{d}_{Oz}-r\textbf{d}_{y}) \\
  &            &&\phantom{{}={}} + v_{Ox}\textbf{d}_{x} + (v_{Oy}+w_{z}r)\textbf{d}_{y} + (v_{Oz}-w_{y}r)\textbf{d}_{z} \\
  &            &&= w_{x}\textbf{d}_{Ox} + w_{y}\textbf{d}_{Oy} + w_{z}\textbf{d}_{Oz} + v_{Ox}\textbf{d}_{x} + v_{Oy}\textbf{d}_{y} + v_{Oz}\textbf{d}_{z} \\
  &            &&= \widehat{v}
\end{alignat*}
}
{}

Nous avons ainsi montré l'équivalence des deux vecturs spatiaux $\widehat{v}$ et $\widehat{v'}$.

\subsection{invariance entre deux bases de \emph{Plücker} quelconques}


