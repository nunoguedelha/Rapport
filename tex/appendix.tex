
\chapter{Notations en Algèbre spatiale} \label{appx_notations}

\chapter{Concepts et théorèmes en mécanique du solide à la base de l'algèbre spatiale} \label{appx_torseursToalgSpa}

\section{Les torseurs} \label{appx_torseursToalgSpa_torseurs}

\subsection{Définition} \label{appx_torseursToalgSpa_torseurs_def}
Les théorèmes généraux de la dynamique des systèmes matériels mettent en jeux deux ensembles de vecteurs liés: celui des quantités de mouvement (cinématique) et celui des forces (dynamique). Ces ensembles n'interviennent que par les torseurs qui leur sont associés. On appelle torseur ($T$) l'ensemble d'un champ antisymétrique $M(A)$ et de son vecteur $S$. $M(A)$ et $S$ sont appelés respectivement \emph{moment} et \emph{vecteur} du torseur $[T]$.

\vspace{0.3cm} % retour à la ligne

\minipages[2]
{.3}{.7}{}
{%
\begin{equation*}
[\underline{T}]=
\begin{bmatrix}
  \underline{S} \\
  \underline{M}(O)
\end{bmatrix}
\end{equation*}
}
{%
\begin{equation}
M(A)=M(B)+AB \times S \textnormal{ ou } M(B)=M(A)+S \times AB
\end{equation}
}
{}

\vspace{0.3cm} % retour à la ligne

Le champ antisymétrique en question n'est pas forcément un champ de vecteurs liés. C'est le cas, par exemple, du torseur cinématique d'un solide en mouvement, constitué à partir du champ antisymétrique des vitesses du solide (vitesse d'un point du solide coïncidant avec un point fixe de l'espace). nous développons ce point dans la section \ref{appx_torseursToalgSpa_torseurs_appl}.

\subsection{Propriétés} \label{appx_torseursToalgSpa_torseurs_prop}

Nous regroupons dans cette section les propriétés des torseurs qui sont directement héritées et utilisées en algèbre spatiale.

\begin{alignat*}{4}
  & \textbf{\textsc{é}galité:} \qquad
    && [T_{1}]=[T_{2}] && \implies && \textbf{S}_{1}=\textbf{S}_{2} \textnormal{ et } \textbf{M}_{1}(A)=\textbf{M}_{2}(A) \\
  & \textbf{Somme:} \qquad
    && [T]=[T_{1}]+[T_{2}] && \implies && \textbf{S}=\textbf{S}_{1}+\textbf{S}_{2} \textnormal{ et } \textbf{M}(A)=\textbf{M}_{1}(A)+\textbf{M}_{2}(A) \\
  & \textbf{Multiplication par un scalaire:} \qquad
    && [T_{1}]=\lambda [T_{2}] && \implies && \textbf{S}_{1}=\lambda \textbf{S}_{2} \textnormal{ et } \textbf{M}_{1}(A)=\lambda \textbf{M}_{2}(A) \\
  & \textbf{Torseur nul:} \qquad
    && \textbf{S}=0 \textnormal{ et } \textbf{M}(A)=0 \\
  & \textbf{Produit scalaire:} \qquad
    && si [F]={\textbf{F},\textbf{M}(A)} \textnormal{ et } [v]={\textbf{w},\textbf{V}(A)} && \implies && [F][v] = \textbf{F} \cdot \textbf{v}(A) + \textbf{M}(A) \cdot \textbf{w} \\
  & \textbf{Invariants:} \qquad
    && \textnormal{la projection de $\textbf{M}$ sur $\textbf{S}$} \\
  & && \textnormal{la projection de $\textbf{M}$ sur $\textbf{AB}$} \\
  & && \textnormal{Le produit scalaire $[F][v]$.} \\
  & \textbf{Torseur lié à un ensemble de vecteurs liés:} \quad
    && \textnormal{Le moment d'un ensemble de vecteurs liés ${(A_{i},\textbf{V}_{i})}$ a la forme d'un champ antisymétrique \cite{bib_champsVecteurs} (chapitre IV). On lui associe donc le torseur} \\ %reprendre ici
  & && && \textbf{S}=\Sigma_{i}\textbf{V}_{i} (\emph{vecteur}) \textnormal{ et } \textbf{M}(O)=\Sigma_{i}\textbf{OA}_{i} \times \textbf{V}_{i} (\emph{moment}).
\end{alignat*}


\subsection{applications} \label{appx_torseursToalgSpa_torseurs_appl}

\subparagraph{Torseur cinématique}

Dans l'étude de la dynamique des systèmes matériels, le champ de vitesses d'un corps en mouvement \footnote{on se limite ici au cas des corps rigides} a la forme d'un champ antisymétrique (vecteurs non liés au corps), et son vecteur \emph{w} définit la vitesse angulaire de rotation du solide. On lui associe donc un torseur dont les éléments de réduction sont:
\begin{description}
\item[moment $M(O)$:] vitesse $\textbf{v}(O)$ du solide au point O
\end{description}

\subparagraph{Torseur dynamique}



\chapter{Quelques démonstrations en algèbre spatiale} \label{appx_dem}

\section{Invariance du vecteur spacial $\widehat{v}$}

\subsection{invariance entre deux bases de \emph{Plücker} ayant la même orientation}

Nous allons démontrer l'invariance du vecteur spatial dans le cas simple illustré ci-dessous:

\dispThreeFig[H]
{4}{Deux bases de \emph{Plücker} de même orientation.}
{5}{Expression de $\textbf{d}_{Py}$ et $\textbf{d}_{Pz}$ en $O$.}
{6}{cas de $\textbf{d}_{Pz}$ (projection suivant $\textbf{d}_{z}$).}
{Expression de $\textbf{d}_{Px}$, $\textbf{d}_{Py}$ et $\textbf{d}_{Pz}$ en $O$.}
{vitessePointCoincidant}

On considère les deux bases de \emph{Plücker}:

\begin{align*}
D_{O} = \lbrace &\textbf{d}_{Ox}, \textbf{d}_{Oy}, \textbf{d}_{Oy}, \textbf{d}_{x}, \textbf{d}_{y}, \textbf{d}_{z} \rbrace \\
D_{P} = \lbrace &\textbf{d}_{Px}, \textbf{d}_{Py}, \textbf{d}_{Py}, \textbf{d}_{x}, \textbf{d}_{y}, \textbf{d}_{z} \rbrace \\
\end{align*}

Dans la section \ref{algSpa_Vitesse}, figure \ref{fig_transPlucker}, nous avons montré la relation entre les vecteurs unitaires de mouvement des deux bases de \emph{Plücker} $D_{O}$ et $D_{P}$:

\begin{align*}
  \textbf{d}_{Px} &= \textbf{d}_{Ox} \\
  \textbf{d}_{Py} &= \textbf{d}_{Oy}+r \textbf{d}_{z} \\
  \textbf{d}_{Pz} &= \textbf{d}_{Oz}-r \textbf{d}_{y}
\end{align*}

On considère à présent un corps $C$ en translation et en rotation autour d'un axe quelconque dans l'espace. Le vecteur spatial $\widehat{v}$ de ce corps en $O$, s'exprime dans la base de \emph{Plücker} $D_{O}$ comme suit:

\begin{equation*}
  \widehat{v} = w_{x}\textbf{d}_{Ox} + w_{y}\textbf{d}_{Oy} + w_{z}\textbf{d}_{Oz} + v_{Ox}\textbf{d}_{x} + v_{Oy}\textbf{d}_{y} + v_{Oz}\textbf{d}_{z}
\end{equation*}

Le vecteur spatial $\widehat{v'}$ de ce corps en $P$, s'exprime dans la base de \emph{Plücker} $D_{P}$ comme suit:

\begin{equation*}
  \widehat{v'} = w_{x}\textbf{d}_{Px} + w_{y}\textbf{d}_{Py} + w_{z}\textbf{d}_{Pz} + v_{Px}\textbf{d}_{x} + v_{Py}\textbf{d}_{y} + v_{Pz}\textbf{d}_{z}
\end{equation*}

Nous voulons l'exprimer dans la base $D_{O}$. $w_{x}$, $w_{y}$ et $w_{z}$ sont connus. Calculons $v_{Px}$, $v_{Py}$ et $v_{Pz}$:

\minipages[2]{.4}{.6}{}
{%
\begin{alignat*}{3}
  & &&\phantom{xxx} v_{P} &&= v_{O} + w \times \overrightarrow{OP} \\
  \vspace{0.5cm}
  &\iff &&\phantom{xxx} \underline{v}_{P}
  &&=
  \underline{v}_{O}+\underline{w}\times
  \begin{bmatrix}
    r\\
    0\\
    0
  \end{bmatrix} \\
  \vspace{0.5cm}
  &\iff
  &&\begin{bmatrix}
    v_{Px} \\
    v_{Py} \\
    v_{Pz}
  \end{bmatrix}
  &&=
  \begin{bmatrix}
    v_{Ox}\\
    v_{Oy}+w_{z}r\\
    v_{Oz}-w_{y}r
  \end{bmatrix}
  \quad \texttt{or} \quad
  \widehat{\underline{v}'}
  =
  \begin{bmatrix}
    \underline{w} \\
    \underline{v}_{P}
  \end{bmatrix}
\end{alignat*}
}
{%
\begin{alignat*}{2}
  &\iff
  \widehat{v'} &&= w_{x}\textbf{d}_{Px} + w_{y}\textbf{d}_{Py} + w_{z}\textbf{d}_{Pz} \\
  &            &&\phantom{{}={}} + v_{Px}\textbf{d}_{x} + v_{Py}\textbf{d}_{y} + v_{Pz}\textbf{d}_{z} \\
  &            &&= w_{x}\textbf{d}_{Ox} + w_{y}(\textbf{d}_{Oy}+r\textbf{d}_{z}) + w_{z}(\textbf{d}_{Oz}-r\textbf{d}_{y}) \\
  &            &&\phantom{{}={}} + v_{Ox}\textbf{d}_{x} + (v_{Oy}+w_{z}r)\textbf{d}_{y} + (v_{Oz}-w_{y}r)\textbf{d}_{z} \\
  &            &&= w_{x}\textbf{d}_{Ox} + w_{y}\textbf{d}_{Oy} + w_{z}\textbf{d}_{Oz} + v_{Ox}\textbf{d}_{x} + v_{Oy}\textbf{d}_{y} + v_{Oz}\textbf{d}_{z} \\
  &            &&= \widehat{v}
\end{alignat*}
}
{}

Nous avons ainsi montré l'équivalence des deux vecturs spatiaux $\widehat{v}$ et $\widehat{v'}$.

\subsection{invariance entre deux bases de \emph{Plücker} quelconques}


