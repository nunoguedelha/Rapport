
% acronymes

\newacronym{rneaLabel}{RNEA}{Recursive Newton-Euler Algorithm}
\newacronym{crbaLabel}{CRBA}{Composite Rigid Body Algorithm}
\newacronym{abaLabel}{ABA}{Articulated Body Algorithm}

%% définitions

\newglossdef{vecteurLie}
{vecteur lié}
{Un vecteur lié $(A,\textbf{V})$ est l'ensemble d'un point $A$ et d'un vecteur $\textbf{V}$ associé à $A$. Par exemple une force ou une vitesse appliquées à un point fixe d'un solide.}

\newglossdef{champVecteurs}
{champ de vecteurs}
{On appelle champ de vecteurs l'application qui fait correspondre, à tout point $A$ de \varepsilon, un vecteur $U$ d'un espace vectoriel $F$ de même dimension que \varepsilon. Par exemple les champs électrique $E$ et magnétique $B$ liés à une charge électrique \cite{bib_champVecteurs}.}

\newglossdef{champAntiSym}
{champ antisymétrique}
{Un champ de vecteurs est \emph{antisymétrique} s'il existe un vecteur $S$ tel que, quels que soient les points $A$ et $B$, on ait: $M(A)=M(B)+AB \times S$. Le vecteur $S$ est \emph{le vecteur} du champ antisymétrique \cite{bib_champVecteurs}.}

\newglossdef{torseur}
{torseur}
{un outil mathématique utilisé principalement en mécanique du solide, pour décrire leurs mouvements et les actions mécaniques qu'ils subissent. Un \emph{torseur} $[T]$ est l'ensemble d'un champ antisymétrique $M(A)$ et de son vecteur $S$. $M(A)$ et $S$ sont appelés respectivement \emph{moment} et \emph{vecteur} du torseur $[T]$.}

